\section{Reaction Times}
\label{sec:reaction-times}

The common assertion is that a player's  reaction  time is 18 frames, and thus,
any move that has a startup time longer than 18  frames  is  always  reactable.

This is \textbf{quite wrong} for three reasons.

First, you can't always recognize  or distinguish what is happening on the very
first      frame      of       a       move.      For      example,      figure
\ref{fig:reaction-times:ledge-options-frame-1}  shows the first frame of  three
different  ledge options. Can you tell which is which? You can't,  because  the
animations only  start  differing  many  frames later. While you may be able to
react to the start  of an animation within 18 frames, you won't be able to tell
which option it is until much later.

Secondly, there are actually two different types of reactions. 18 frames refers
to the ``Single Response Time (SRT)'' reaction. You have already predicted what
they are going to do, but you don't know when they are going to do it. A single
stimulus  triggers your response, and this response  time  is  said  to  be  18
frames.

The second type of reaction is called ``Choice Response Time (CRT)''. Here, you
neither  know  what they are going to do nor when they are going to do it. This
type  of reaction takes much longer -- around 30 frames, in fact -- because you
have make a \emph{choice} on top of reacting to the initial stimulus of when to
make the decision.

\begin{figure}[ht]
    \centering
    \begin{subfigure}[t]{0.48\linewidth}
        \resizebox{\linewidth}{!}{\digraph{srt}{
            pad=0;
            rankdir=LR;
            subgraph {
                a [shape="record", label="Situation"];
                b [shape="record", label="Response"];
                a->b [label="Stimulus"];
            }
        }}
        \caption{Single Response Time (SRT)}
    \end{subfigure}
    \begin{subfigure}[t]{0.48\linewidth}
        \resizebox{\linewidth}{!}{\digraph{crt}{
            pad=0;
            rankdir=LR;
            subgraph {
                a [shape="record", label="Situation"];
                b [shape="record", label="Response 1"];
                c [shape="record", label="Response 2"];
                a->b [label="Stimulus 1"];
                a->c [label="Stimulus 2"];
            }
        }}
        \caption{Choice Response Time (CRT)}
    \end{subfigure}
\end{figure}
